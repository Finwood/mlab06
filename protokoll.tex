\documentclass[a4paper,12pt]{scrartcl}
\usepackage[utf8]{inputenc}
%\usepackage[T1]{fontenc}
\usepackage[ngerman]{babel}
\usepackage[babel]{csquotes}
\usepackage{fancyhdr}
\usepackage{amssymb}
\usepackage[fleqn]{amsmath}
\usepackage[a4paper, left=30mm, right=25mm, bottom=25mm, top=30mm]{geometry}
\usepackage{tipa}
\usepackage{titlesec}
\usepackage{graphicx}
\usepackage[labelfont=bf]{caption}
\usepackage{subcaption}
\usepackage{float}
\usepackage{tocloft}
\usepackage{booktabs}
\usepackage{multirow}
\usepackage{multicol}
\usepackage{array}
\usepackage[exponent-product = \cdot]{siunitx}
\usepackage{cancel}
\sisetup{per-mode=fraction}

%\usepackage{pythontex}
\graphicspath{ {.} {img/} }

\usepackage[usenames, divpsnames]{xcolor}
\definecolor{footer-gray}{gray}{0.5}

\title{Versuchstitel}
\date{2014-11-05}

\renewcommand{\cftsecdotsep}{\cftdotsep}

\pagestyle{fancy}
\linespread{1.3} % entspricht 1.5-fachem Zeilenabstand
\setlength{\parskip}{10pt}
\setlength{\parindent}{0mm}
\setlength{\mathindent}{10mm}

% enable draft mode --> revision is being shown
\newcount\draft\draft=1

\ifnum\draft=1
	\IfFileExists{revision.tex}{
		\include{revision}
	}{
		\newcommand{\Revision}{}
	}
\fi

\lhead{Versuchstitel}
\chead{}
\rhead{\thepage}

\ifnum\draft=1
	\lfoot{\textcolor{footer-gray}{\today; rev. \Revision}}
\else
	\lfoot{}
\fi
\cfoot{}
\rfoot{}

% Page Numbering
%\setcounter{tocdepth}{3}
\usepackage{hyperref}
\hypersetup{
    colorlinks,
    citecolor=black,
    filecolor=black,
    linkcolor=black,
    urlcolor=black
}

%\usepackage[backend=biber, style=alphabetic-verb, sortlocale=de_DE, natbib=false, url=true, doi=true, eprint=true]{biblatex}
%\addbibresource{sources.bib}

\begin{document}

\thispagestyle{empty}
\newgeometry{left=3cm, right=3cm}
\begin{center}
\vspace*{20mm}

\Large
\textbf{Grundpraktikum Maschinenlabor} \\
\vspace{10mm}
\Huge
\textbf{\textsf{Versuchstitel}} \\
\vspace{10mm}
\large
\textbf{\textsf{Praktikumsbericht}} \\
\textbf{\textsf{Mittwoch, x. November 2014}} \\
\textbf{\textsf{Gruppe 6}} \\
\ifnum\draft=1
	\textsf{\textcolor{footer-gray}{Version \textit{\Revision} vom \today}} \\
\fi

\vspace{40mm}
\begin{tabular}{rcl}
Ole Brinkmann & 406572 & obr11@tu-clausthal.de \\[4mm]
Lasse Fröhner & 420013 & lf12@tu-clausthal.de \\[4mm]
Sebastian Löhr & 341792 & swl@tu-clausthal.de \\[4mm]
Anna Stillfried & 421777 & apmsr12@tu-clausthal.de \\[4mm]
\end{tabular}

\end{center}

\pagebreak
\restoregeometry

\addtocontents{toc}{\protect\thispagestyle{fancy}}
\addtocontents{lof}{\protect\thispagestyle{fancy}}
\pagenumbering{Roman}
\tableofcontents
%\vspace{20mm}
%\listoffigures
%\vspace{20mm}
%\listoftables
\newpage

\pagenumbering{arabic}

\section{Einleitung}

\section{Aufbau}

\section{Durchführung}

\section{Auswertung}

\section{Fazit}

%\pagebreak
%\nocite{*}
%\sloppy % großzügige Wortabstände zulassen --> URLs werden umgebrochen
%\printbibliography


\end{document}

